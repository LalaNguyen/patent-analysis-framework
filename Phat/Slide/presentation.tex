\documentclass{beamer}
\mode<presentation>{
\usetheme{Madrid}
}
\usepackage{graphicx}
\usepackage{booktabs}
\usepackage{multicol}

%--------------------------------------------------
%	TITLE PAGE
%--------------------------------------------------
\title[Citation Network Proposal]{Citation Network Proposal}

\author[Group 2 - ADB Course]{Group 2\\MITIU14005 - Nguyen Hoang Minh \\ MITIU14006 - Chiem Thach Phat}

\institute[HCMIU]{Ho Chi Minh City, International University}

\date{\today}

%--------------------------------------------------
\begin{document}
%--------------------------------------------------
\begin{frame}
\titlepage
\end{frame}

\begin{frame}
\frametitle{Table of Contents}
\tableofcontents
\end{frame}
%--------------------------------------------------
\section{Introduction}
%--------------------------------------------------
\begin{frame}
\frametitle{Table of Contents}
\tableofcontents[currentsection]
\end{frame}

\begin{frame}
\frametitle{Motivation}
The world need a citation network for patents to:
\begin{itemize}
\item Analyze/Predict competitors trend
\item Look up whether there is a similar patent or not to submit a new one
\item Find who invent most patents
\item Check uniqueness of company's products
\item Find which products or which patents are most cited by other
\end{itemize}
\end{frame}

%--------------------------------------------------
\section{Implementation}
%--------------------------------------------------
\begin{frame}
\frametitle{Table of Contents}
\tableofcontents[currentsection]
\end{frame}

\begin{frame}
\frametitle{Scenario}
We want to build a citation network for patents with these constraints:
\begin{itemize}
\item Able to store and analyze patents from USPTO
\item A patent contains: an unique Publication Number, Publication Type, Application Number, Publication Date, Filling Date, Priority Date, Fee Status, Title, Length of Grant, Classification and some of its contents include Abstract, Images, Description and Claims
\item A Patent may or may not cite/be cited other patents
\item A Patent must be invented/applied/examined by or assigned for at least one Inventor, Applicant, Examiner or Assignee
\item Inventor, Applicant, Examiner and Assignee are identified by first name, last name, city and country
\end{itemize}
\end{frame}

\begin{frame}
\frametitle{Conceptual Model}
Based on the scenario, we have design this conceptual model.
\begin{center}
\includegraphics[height=2in]{full_er.png}
\end{center}
\end{frame}

\begin{frame}
\frametitle{Implemented Conceptual Model}
Due to time constraint, we use this conceptual model to implement the citation network.
\begin{center}
\includegraphics[height=2in]{implemented_er.png}
\end{center}
\end{frame}

\begin{frame}
\frametitle{Logical Model - Neo4j}
Neo4j use the concepts of \textbf{Nodes}, \textbf{Properties} and \textbf{Relationships} as shown in the picture below, where:
\begin{itemize}
\item Nodes: PATENT, INVENTOR
\item Properties: Nodes and Relationships attributes, such as Patent\_No or Cited\_date
\item Relationships: CITEDBY, INVENT
\end{itemize}
\begin{center}
\includegraphics[height=1.8in]{neo4j.png}
\end{center}
\end{frame}

\begin{frame}
\frametitle{Logical Model - Neo4j (Cont.)}
Pros:
\begin{itemize}
\item Can cope with complex and highly connected data
\item Offers more accurate ways to determine the relationship between each entity
\end{itemize}
Cons:
\begin{itemize}
\item Slow with complex search
\item Server is not supported for visualizing more than 1000 rows
\end{itemize}
\end{frame}

\begin{frame}
\frametitle{Logical Model - MongoDB}
While Neo4j stores data in Nodes, Properties and Relationships, \textbf{MongoDB stores data as a BSON file}.
\begin{center}
\includegraphics[height=2.6in]{mongodb.png}
\end{center}
\end{frame}

\begin{frame}
\frametitle{Logical Model - MongoDB (Cont.)}
Pros:
\begin{itemize}
\item Documents (i.e. objects) can be defined with native data types that can be seen in many programming languages. E.x: Date, String, Integer
\item Embedded documents and arrays reduce need for expensive joins
\item Dynamic schema supports fluent polymorphism
\end{itemize}
Cons:
\begin{itemize}
\item Does not support identify duplication
\end{itemize}
\end{frame}

\begin{frame}
\frametitle{CINAS Structure}
Our CINAS will follow the structure below, starts from the Patent XML File and ends at Neo4j DB and MongoDB
\begin{center}
\includegraphics[height=2in]{cinas.png}
\end{center}
\end{frame}

\begin{frame}
\frametitle{Grant Parser Module}
Due to the enormous size of the Patent Bulk Download File, we will use \textbf{SAX method} instead of DOM method to \textbf{process the XML file}.\\~\\
Pseudocode for \textbf{Grant Parser Module}:
\begin{enumerate}
\item Load XML file and break it down into separated XML instances
\item Parse each XML instance into patent handler. For each tag with nested tags, enable stack to store adjacent data
\item Tags that are explicitly defines in ignore list will be left out during parsing. Upon completion, reset the patent handler for new instance
\item Finally, an option to export the entire file to JSON format is available if chosen
\end{enumerate}
\end{frame}

\begin{frame}
\frametitle{DB Client Module}
We chose to deploy REST services in communication with our backends instead of OM to provide more scalable solution as we just use HTTP response and request.\\~\\
\textbf{DB Client Class Diagram:}\\
\begin{center}
\includegraphics[height=2in]{dbclient.png}
\end{center}
\end{frame}

\begin{frame}
\frametitle{DB Client Module (Cont.)}
Pseudocode for \textbf{POST in MongoDBConnector}:
\begin{enumerate}
\item Load all data from JSON file
\item Construct list of citations from group of patent
\item Validate uniqueness within the list
\item Appending statements into query. Then, perform 1st POST these citations
\item If the uniqueness is violated, the response code from EVE server will determine the next step. If error response with duplicates found, remove it from the list and resend it again
\item Finally, perform an overwrite PUT for any individual patent. This step must be linearly execute to guarantee that all the dummy patents, which are created using citations, will be updated with information
\end{enumerate}
\end{frame}

\begin{frame}
\frametitle{DB Client Module (Cont.)}
POST in Neo4j is more simpler because it has an overwrite operation called \textbf{``MERGE''}, which \textbf{automatically detect duplication and merge together or create new one}.\\~\\
Pseudocode for \textbf{POST in Neo4jDBConnector}:
\begin{enumerate}
\item Load all data from JSON file
\item Construct list of citations from group of patent
\item Appending statements into query. Then, perform POST these citations
\item Finally, we perform another POST for list of patents
\end{enumerate}
\end{frame}

\begin{frame}
\frametitle{Data Visualization}
Data Visualization Process:
\begin{enumerate}
\item Submit a query to database server
\item Receive a JSON Response
\item Generate CSV File from JSON Response
\item Use GEPHI to load CSV and visualize data
\end{enumerate}
\begin{center}
\includegraphics[height=2in]{gephi.png}
\end{center}
\end{frame}

%--------------------------------------------------
\section{Results}
%--------------------------------------------------
\begin{frame}
\frametitle{Table of Contents}
\tableofcontents[currentsection]
\end{frame}

\begin{frame}
\frametitle{Results}
Processed patents:
\begin{table}[htd]
\begin{tabular}{|c|c|l|} \hline
 &Number of Documents\\ \hline
 Duplicated Citations&347 \\ \hline
Patents&3000 \\ \hline
Citations&106996 \\ \hline
Total & 110343\\ \hline
\end{tabular}
\end{table}
\end{frame}

\begin{frame}
\frametitle{Results (Cont.)}
The figure shows Patent Classification in October 2014.\\~\\
The MISCELLANEOUS (\textbf{D99}) industry were blossomed in October 2014
\begin{center}
\includegraphics[height=2in]{rplot-classification.png}
\end{center}
\end{frame}

\begin{frame}
\frametitle{Citation Network}
The figure shows a citation network with 500 patents in the database.\\~\\
D0714930, D0714929 and D0714813 were the three most cite other patents corresponding to their node's radius.
\begin{center}
\includegraphics[height=2.6in]{citation-label.png}
\end{center}
\end{frame}

\begin{frame}
\frametitle{Inventors Maps}
To measure the activity of inventors community, we classify them based on their regions.\\~\\
Most of inventors come from North America. Furthermore, US patents industry also attracts the numerous inventors from different parts of the world, including Asian and Australian.
\begin{center}
\includegraphics[height=2in]{inventor-map.png}
\end{center}
\end{frame}

\begin{frame}
\frametitle{Benchmark Neo4j and MongoDB}
We perform the time measurements on 500 patents with total of 12016 citations.\\
\begin{multicols}{2}
Results:
\begin{itemize}
\item \textbf{DOM} parser is \textbf{faster} than \textbf{SAX}
\item \textbf{OGM} is \textbf{faster} than \textbf{REST}, but \textbf{REST} is \textbf{simpler} than OGM
\item Populate data in \textbf{MongoDB} is \textbf{faster} than \textbf{Neo4j}, but the drawback is MongoDB client has to implement references between documents
\end{itemize}
\columnbreak
\begin{center}
\includegraphics[width=0.5\textwidth]{line.png}
\end{center}
\end{multicols}
\end{frame}

%--------------------------------------------------
\section{Future works}
%--------------------------------------------------
\begin{frame}
\frametitle{Table of Contents}
\tableofcontents[currentsection]
\end{frame}

\begin{frame}
\frametitle{Future Works}
\begin{itemize}
\item Divide patent data into different databases instead of storing the whole file into one particular back-end
\item Propose a refined structure for storing and querying patent data across multiple database systems
\item Provide an end-user services that would transform natural language into our CINAS's language, process the queries, determine the keywords and then pass it to the suitable database system
\item Final goals:
\begin{enumerate}
\item Take advantages of multiple database systems
\item Reduce the query costs
\item Enhance end-users experience with patent data
\end{enumerate}
\end{itemize}
\end{frame}

\begin{frame}
\frametitle{References}
[1] ``Transitioning from Relational Databases to MongoDB - Data Models'' B. Reinero, http://blog.mongodb.org/post/72874267152/transitioning-from-relational-databases-to-mongodb, Jan 2014\par
[2] ``USPTO Patents Database Construction" I. Torvik, D. Dubin and Q.Liu, http://abel.lis.illinois.edu/UPDC/USPTOPatentsDatabaseConstruction.pdf, August 2012\par
[3] ``Construction of Japanese Patent Database for Research on Japanese patenting activities" A. Goto, K. Motohashi, http://www.iip.or.jp/e/images/paper.pdf
\end{frame}

\begin{frame}{Thank You}
\begin{center}
Thank you for your listening!
\end{center}
\end{frame}

\end{document}